\subsection{General discussion}

\subsubsection{Diffusion and movement}
In agent based modeling two different modes for updating can be distinguished:
A synchronous mode updates all cells simultaneously, i.e. local changes are stored in a temporary copy and will be updated after the computation of all cells.
Contrary to this, a asynchronous mode updates changes immediately (\cite{Matthies2002} p. 92).\\
In our study we implemented for the diffusion rule applied to agents a naive model, which relies on the asynchronous update with randomly chosen cells.
For the diffusion of substrates this method is preferred, because i) synchronous updates would violate the conservation laws by the production of additional metabolite concentrations and ii) non-random asynchronous updates cause a biased diffusion direction \cite{Bandman1999}.
As indicated in Figure \hyperref[fig:diff]{\ref{fig:diff}} the spreading of metabolite concentrations causes the increase of entropy in the system, which is also observed as a physical phenomenon in microbial communities \cite{Wetzel93}.
Additional refinements can be realized with more sofisticated diffusion models such as block-rotation \cite{Bandman1999} or the discrete diffusion model by Grajdeanu \cite{Grajdeanu2007}.
Different diffusion coefficients of certain metabolites could also be included to model the varying dispersal speeds.\\
The bacterial movement was implemented similar to the diffusion model as a random spread on the grid environment (Figure \hyperref[fig:mov]{\ref{fig:mov}}).
Since most bacteria are able to sense the metabolite concentrations in their environment and show a directed motion to the substrate of choice \cite{Francisco13}, the movement model can be refined by interaction of the bacterial agents with the substrates.
A chemotaxis model would also probably allow the observation of more complex behaviours such as the aggregation to certain parts of the grid environment.

\subsubsection{Time consumption}
FBA calculation of substrate exchange can take $2\,min$ per time step for $180$ bacteria of the big \textit{E. coli} model ($1972\times 2382$ metabolites and substrates).
This leads to an overall calculation time of several hours for the \textit{Bcoli} population model.
In comparison to this, the \textit{E. coli} ($77\times 77$) core model needed only $2\,$s per time step for $180$ bacteria.
Almost all of the required time is spent for fba calculation.
For this reason further improvements can be done by:
\begin{enumerate}
  \item Hashing: Implementation of a fba memory table, which saves prior calculations.
  \item Starting base: \textit{lpsolve} offers since version $5.1.05$ the possibility to find a basis according to some guessed vector (old solution).
    With this basis the optimal solution of the optimization problem might be found faster \cite{warmstart}.
\end{enumerate}
We implemented already a first form of hashing, however the speed improvements were so far not quantified and compared. 

\subsection{Population models of single organisms}
All population models show similarities in their growth curves (Figure \hyperref[fig:ecoresg]{\ref{fig:ecoresg}}, \hyperref[fig:ecolisg]{\ref{fig:ecolisg}}, \hyperref[fig:barkerisg]{\ref{fig:barkerisg}} and \hyperref[fig:beijersg]{\ref{fig:beijersg}}).
Roughly, the observed growth can be separated into three phases: the exponential, the stationary and the death phase. Those phases are also observed in experimental studies of microbial growth \cite{Varma1994}.
During the exponential phase all substrates are efficiently used by the fba to accumulate biomass, which is then used for duplication. In the stationary phase almost all substrates were exploited and the fba does not find any feasible solution, which results in the reduction of biomass. Subsequently, bacterial agents are removed, if the biomass is below zero, leading to the death phase.

According to experimental studies \cite{Varma1994}, the time for \emph{E. coli} to reach the stationary phase is about 8\;h, which is in contradiction to the observed time of approximately 39\;h in our study (Figure \hyperref[fig:ecoresg]{\ref{fig:ecoresg}}). This can be explained by the used artificial media composition, which might not be sufficient for an optimal growth.
Moreover, the substrate concentrations were set to not empirically justified arbitrary values. Further refinements of substrate concentrations could thus increase the accuracy of the model to match experimental results. Additionally, the constraints of the used exchange reactions can also be refined to more realistic values.

The observed growth of \emph{E. coli} (Figure \hyperref[fig:ecoresg]{\ref{fig:ecoresg}}) can be explained mainly by the aerobic respiration of glucose:
\[
  \textrm{C}_{6}\textrm{H}_{12}\textrm{O}_{6} + 6\textrm{O}_2 \rightarrow 6\textrm{CO}_{2} + 6 \textrm{H}_{2}\textrm{O}
\]
where one molecule of glucose and 6 molecules of oxygen are consumed to produce 6 molecules of CO$_2$ and water. In accordance to the stochiometry of this reaction more oxygen was consumed by the model compared to glucose (Figure \hyperref[fig:ecolisg]{\ref{fig:ecolisg}}). Additionally, various fermentation products are produced via mixed acid fermentation under aerobic conditions (Figure \hyperref[fig:ecoresg]{\ref{fig:ecoresg}}), which is in concordance to experimental studies \cite{Sunya13}.
In these studies, formate is only produced under anaerobic conditions. Interestingly, the production of formate is observed under aerobic conditions in \emph{E. coli} core, whereas \emph{E. coli} big does not produce formate (Figure \hyperref[fig:ecoresg]{\ref{fig:ecoresg}}).
Furthermore, the smaller \emph{E. coli} core model produces overall more fermentation products relative to CO$_2$. This can be explained by the additional pathways present in the more representative larger \emph{E. coli} model, which make the complete mineralization of glucose to CO$_2$ more preferable in the fba optimization.
Since the complete mineralization of glucose would be even more preferable in the larger \emph{E. coli} model, the applied uptake constraints (Table \hyperref[ab:const]{\ref{tab:const}}) were set to allow the production of fermentation products.

The growth of \emph{M. barkeri} (Figure \hyperref[fig:barkerisg]{\ref{fig:barkerisg}}) can be explained by the fermentation of methanol to methane with
\[
  4\textrm{CH}_{3}\textrm{OH} \rightarrow 3\textrm{CH}_{4} + \textrm{CO}_{2} + 2\textrm{H}_{2}\textrm{O}
\]
where 4 molecules of methanol are consumed to produce 3 molecules of methane, 2 of water and 1 molecule CO$_2$.
The stochiometry of this reaction was fulfilled by the higher production of methane compared to CO$_2$ (Figure \hyperref[fig:barkerisg]{\ref{fig:barkerisg}}). However, water was, similar to methane, produced in high amounts, which can be explained by additional reactions in the model, which might lead to water production.
In experimental studies the stationary phase of \emph{M. barkeri} under the consumption of methanol is reached after 72\;h \cite{Hippe79} which is in concordance to our observed time of approximately 80\;h. 

\textit{C. beijerinckii} is able to convert glucose to various fermentation products \cite{Ezeji03} via butyrate fermentation. To remove reducing equivalents hydrogen is formed, which is also observed in our model (Figure \hyperref[fig:beijersg]{\ref{fig:beijersg}}). The time for \textit{C. beijerinckii} to reach the stationary phase with glucose as a substrate was experimentally determined to approximately 60\;h \cite{Ezeji03}, which is in contradiction to the observed time of 40\;h (Figure \hyperref[fig:beijersg]{\ref{fig:beijersg}}). To match the experimental results the constraints of the exchange reaction can be refined.

Growth deviations (especially doubling time) have to be checked with multiple runs and a wide range of circumstances to guarantee more representative results.

\subsection{Interactions in mixed communities}
Since \textit{M. barkeri} can also utilize hydrogen and CO$_2$ for methane production and \textit{C. beijerinckii} is able to produce those metabolites, both organisms are good candidates to study inter-species hydrogen transfer and syntrophy.
Furthermore, co-culture studies on \textit{M. barkeri} with hydrogen producing microbes \cite{Winter79} have demonstrated the capabilities of \textit{M. barkeri} to interact with other organisms.

In our mixed community model (Figure \hyperref[fig:cbsg]{\ref{fig:cbsg}}) we could observe the usage of \textit{C. beijerinckii}'s produced hydrogen and CO$_2$ by \textit{M. barkeri}. However, in contrast to experimental observations showing the parallel growth of both organisms \cite{Winter79}, we observed a delayed growth of \textit{M. barkeri} after the death of \textit{C. beijerinckii} (Figure \hyperref[fig:cbsg]{\ref{fig:cbsg}}). This can be explained by the limited space on the grid environment, which favoured the initial growth of \textit{C. beijerinckii}, since enough substrates were available. Consequently, \textit{M. barkeri} did not have enough space to duplicate and could only spread after the other organism died out (Figure \hyperref[fig:cbgrid]{\ref{fig:cbgrid}}). 
Nevertheless, \textit{M. barkeri} was still able to survive, due to the hydrogen and CO$_2$ support by \textit{C. beijerinckii}.
With the production of \textit{M. barkeri}'s essential metabolites, \textit{C. beijerinckii} thus shaped the environment of the other organism.

In the co-culture of \textit{C. beijerinckii} with \textit{E. coli} we observed competitive interactions, where both organisms competed for the same substrate. In this context, \textit{C. beijerinckii} was able to outcompete the other bacterium with an overall higher growth (Figure \hyperref[fig:cegrid]{\ref{fig:cegrid}}). \textit{E. coli} was not able to show high growth rates, since the main substrate was consumed by the competitor.

These results might indicate a more efficient anaerobic degradation by \textit{C. beijerinckii} compared to \textit{E. coli}.
Another explanation can be the pure chance of \textit{C. beijerinckii} having better starting conditions, which lead to a higher growth (positive feedback).
Further experimental co-culture studies might validate this observation.

In the co-culture of \textit{M. barkeri} with \textit{E. coli} we observed space competitive effects, in which the space was not sufficient to harbour both organisms at their stationary phase (Figure \hyperref[fig:begrid]{\ref{fig:begrid}}). The growth curves of both species were similar to each other, which speaks for no competitive metabolic interactions as observed in the \textit{C. beijerinckii}, \textit{E. coli} joint model (Figure \hyperref[fig:cesg]{\ref{fig:cesg}}).
Moreover, \textit{M. barkeri} was able to utilize acetate produced by \textit{E. coli} (Figure \hyperref[fig:besg]{\ref{fig:besg}}).

\subsection{Conclusions \& outlook}
In the last years a lot of work has been done to build up models for single organisms \cite{lewis2012}.
This models are capable to reproduce experimental results and permit even quantitative phenotypic predictions \cite{mccloskey}.
Recently, voices were being raised to point out the future direction of research on community systems biology:
\begin{quote}
,,We anticipate that, through the use of bottom-up approaches supplemented with meta-omics data, the success of systems biology for individual organisms will now be extended to communities of organisms, and in particular to microbial communities.'' \cite{cosys}
\end{quote}
We implemented such a bottom up approach in our agent based model, which uses the established strength of individual models to simulate metabolic interactions and even syntrophy between communities of \textit{in silico} species.
