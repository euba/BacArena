\documentclass{scrartcl}
\usepackage[utf8]{inputenc}
\usepackage{amsmath}
\usepackage{graphicx}
\setlength{\parindent}{0ex}
\usepackage{booktabs}
\usepackage{caption}


\usepackage[
%  style=authoryear,
%  maxcitenames=1,
%  maxbibnames=99
  natbib=true,
  backend=bibtex
]{biblatex}
\addbibresource{literatur.bib}

\setlength{\bibitemsep}{0.5\baselineskip}

\usepackage{longtable}
\usepackage{amssymb}
\usepackage{hyperref}
\usepackage[ruled,noline]{algorithm2e}
\providecommand{\SetAlgoLined}{\SetLine}
\providecommand{\DontPrintSemicolon}{\dontprintsemicolon}
\usepackage[left=2cm,right=2cm,top=1.3cm,bottom=2.2cm]{geometry}

\usepackage{setspace}
\onehalfspacing

\usepackage{subfigure}


\DeclareCaptionFormat{upper}{#1#2\uppercase{#3}\par}
\renewcommand{\thesubfigure}{\Alph{subfigure}}


\newenvironment{items}{
\begin{itemize}
  \setlength{\itemsep}{1pt}
  \setlength{\parskip}{0pt}
  \setlength{\parsep}{0pt}
}{\end{itemize}}

\title{\texttt{BacArena}: Simulation of Interactions in Microbial Communities using Genome-wide Metabolic Reconstructions}
\author{Johannes Zimmermann\\Eugen Bauer}
\date{\today}

\begin{document}

\maketitle

\section*{Abstract}
Microbial communities are essential for global ecosystems and human health.
Computational modeling of microbial consortia is thus a major goal in systems biology and microbial ecology. 

\texttt{BacArena} is a project to simulate bacterial behaviour in communities. A lot of progress is done in the last years to gain genome wide metabolic reconstructions of certain organisms, which open a wide field of mathematical analysis.
One of this new methods is flux balanced analysis (fba) to estimate optimal metabolic fluxes under certain constraints. By this work advanced models are possible, which are available in a defined, exchangeable format (\emph{SBML}). The idea of this project is to use this existing reconstructions and put them in a spatial and temporal environment to study their possible interactions.
This is achieved by the combination of agent based modeling with fba. Each bacterium is considered as an agent with individual states, own properties and rules to act. Agents are located on a grid where they can move and interact via metabolic exchanges computed by fba.

The starting point for our project is curiosity of what could be done with this huge models. We just throw those models into an arena to see what kind of actions will evolve.
\newpage

\tableofcontents

\newpage

\section{Introduction}
\subsection{Microbial metabolic ecology}

\subsection{Constrained based modeling}
Constrained based modeling is one big and succesful development line in systems biology. (\cite{Esvelt2013}, \cite{Klipp2010} p. 353)
Metabolism could be considered as a network of biochemical reaction.
The reaction network itself is accessible to a more formal representation as differential equations using mass action kinetics.
Because of high numbers of reactions and metabolites the corresponding system of equation and the solution space is high dimensional, too.
Linear algebra is used for simplified description:
\[
  \frac{dx}{dt}=S \cdot v
\]
Where $x\in \mathbb{R}^m$ is a vector consisting of concetrations of all $m$ metabolites, $S\in \mathbb{Z}^{m\times r}$ is the stoichiometric matrix, which covers the net comsumption/production of all $r$ biochemical reactions and $v \in \mathbb{R}^r$ symbolised the flux vector and contains therefore the in general nonlinear kinectic relationships.\\
Now several \textit{constraints} could be applied to achive a more easier to solve problem.
The most prominent constraint is equilibrium or steady state $dx/dt \stackrel{!}{=}0$.
It's a reasonable assumption for a metabloic model because there are a lot of evidence for a metabolic steady state in general (i.e. no net change for every metabolites all the time).(\cite{Harris1995} p. 10-11)\\
One important constrained based modeling method is flux balance analysis (fba) (\cite{Varma1994}, \cite{Orth2010}), which will be of the utmost importance for this project.
By this the former nonlinear problem $dx/dt=S\cdot v$ diminishes to $dx/dt=S\cdot v \stackrel{!}{=}0$, which constitutes a normal linear equation systems.
Nevertheless there are far more reactions than metabolites ($r>m$), so that this linear reaction system is underdetermined.
That's why other constraints like flux limits are added.
Flux limits are reaction limits, which narrow down each reaction to some intervall (e.g. irreversible reaction number $i$ has a flux $v_i>0$).
By this the solution space is shrinked.
If the biomass composition and non and growth accociated maintenance (ngam/gam) is known, it is possible to formulate an optimization problem:
\begin{equation*}
  \begin{aligned}
    & \underset{v}{\text{maximize}} & & b(v) \\
    & \text{subject to} & & S \cdot v = 0 \\
    & & & l_i < v_i < u_i
  \end{aligned}
\end{equation*}
where $l_i$ and $u_i$ are the lower and upper limits for reaction $i$ and $b(v)$ is biomass function, which is going to maximized with respect to an certain flux $v$.
Thus we search a vector $v$ carrying quantitative values for all fluxes in the whole raction system so that a certain function (here biomass function) is optimal.


\subsection{Agent based modeling}
\textit{Complexity theory} is a development in system science since 1970s.
Order isn't any longer considered as something given but is itself to be made.
Order should be producable as a surface phenomenon by a complex process, which is i) self organizing, ii) secures its autonomity and iii) proceeds remote from equilibrium.(\cite{Cilliers2007} p. 8-10)\\
According to John Holland, who introduced the important notion of an \textit{agent}, a complex adaptive system (CAS) is defined as follows:
\begin{quote}
,,We will view CAS as systems composed of interacting agents described in terms of rules. These agents adapt by changing their rules as experience accumulates.'' (\cite{Holland1995} p. 10)
\end{quote}
In this modeling paradigm no general differential equation governs the macro behaviour.
The parts of the system are called agents and are explicitly described by \textit{rules} instead of regiours theory.
This enables the possibility to model ,,microscopic'', that is to give individual properties and limited information to agents.
,,If-then rules'' are heuritic and could depict phenomena, where no mathematical description exists.
So far: Agents have individuality, live in a surrounding (grid) with limited radius, so that local interaction in the neighbourhood governed by rules are important.\\
From this microscopic actions the global organization is produced.
New properties and behauviour could occure and this is noted by the slight magical term \textit{emergence}.(\cite{Zimmermann2010} p. 36-39)


\subsection{Aim of the project}



\section{Methods}
\subsection{Model overview}

\subsection{Representation}
\subsubsection{Environment \& Grid}
\subsubsection{Bacteria}
\subsubsection{Substrate}

\subsection{Interactions as rules}
\subsubsection{Movement}
\subsubsection{Diffusion}
\subsubsection{Flux balance analysis}

\subsection{Growth}


\section{Results}
\subsection{Movement and Diffusion}

\begin{figure}[h]
  \centering
  \begin{minipage}[t]{0.3\textwidth}
    \includegraphics[width=\textwidth]{mov1.pdf}
  \end{minipage}
  \begin{minipage}[t]{0.3\textwidth}
    \includegraphics[width=\textwidth]{mov2.pdf}
  \end{minipage}
  \begin{minipage}[t]{0.3\textwidth}
    \includegraphics[width=\textwidth]{mov3.pdf}
  \end{minipage}
  \caption{Bacterial movement starting with a line of bacteria in the middle (timestep 1, 2, 5)}
\end{figure}

\begin{figure}[h]
  \centering
  \begin{minipage}[t]{0.3\textwidth}
    \includegraphics[width=\textwidth]{diff1.pdf}
  \end{minipage}
  \begin{minipage}[t]{0.3\textwidth}
    \includegraphics[width=\textwidth]{diff2.pdf}
  \end{minipage}
  \begin{minipage}[t]{0.3\textwidth}
    \includegraphics[width=\textwidth]{diff5.pdf}
  \end{minipage}
  \caption{Diffusion on a $20\times20$ grid starting with $10\; mmol$ pyruvate in the middle (timestep 1, 2, 5)}
\end{figure}

\subsection{Flux Balance Analysis}
\subsection{Growth models}
\clearpage
\subsubsection{\textit{Escherichia coli} core}

\begin{figure}[h]
  \centering
  \subfigure[]{
    \begin{minipage}[t]{0.3\textwidth}
    \includegraphics[width=\textwidth]{../results/ecoli_20x20_aerob_seed55_bac5.pdf}
  \end{minipage}
  \begin{minipage}[t]{0.3\textwidth}
    \includegraphics[width=\textwidth]{../results/ecoli_20x20_aerob_seed55_bac35.pdf}
  \end{minipage}
  \begin{minipage}[t]{0.3\textwidth}
    \includegraphics[width=\textwidth]{../results/ecoli_20x20_aerob_seed55_bac50.pdf}
  \end{minipage}
  }
  \subfigure[]{
  \begin{minipage}[t]{0.3\textwidth}
    \includegraphics[width=\textwidth]{../results/ecoli_20x20_aerob_seed55_gluc5.pdf}
  \end{minipage}
  \begin{minipage}[t]{0.3\textwidth}
    \includegraphics[width=\textwidth]{../results/ecoli_20x20_aerob_seed55_gluc35.pdf}
  \end{minipage}
  \begin{minipage}[t]{0.3\textwidth}
    \includegraphics[width=\textwidth]{../results/ecoli_20x20_aerob_seed55_gluc50.pdf}
  \end{minipage}
  }
  \subfigure[]{
  \begin{minipage}[t]{0.3\textwidth}
    \includegraphics[width=\textwidth]{../results/ecoli_20x20_aerob_seed55_ace5.pdf}
  \end{minipage}
  \begin{minipage}[t]{0.3\textwidth}
    \includegraphics[width=\textwidth]{../results/ecoli_20x20_aerob_seed55_ace35.pdf}
  \end{minipage}
  \begin{minipage}[t]{0.3\textwidth}
    \includegraphics[width=\textwidth]{../results/ecoli_20x20_aerob_seed55_ace50.pdf}
  \end{minipage}
  }
  \subfigure[]{
  \begin{minipage}[t]{0.3\textwidth}
    \includegraphics[width=\textwidth]{../results/ecoli_20x20_aerob_seed55_co25.pdf}
  \end{minipage}
  \begin{minipage}[t]{0.3\textwidth}
    \includegraphics[width=\textwidth]{../results/ecoli_20x20_aerob_seed55_co235.pdf}
  \end{minipage}
  \begin{minipage}[t]{0.3\textwidth}
    \includegraphics[width=\textwidth]{../results/ecoli_20x20_aerob_seed55_co250.pdf}
  \end{minipage}
  }
  \caption{Aerobic growth of ecoli core model, grid 20x20, seed=55}
\end{figure}

\begin{figure}[h]
  \centering
  \begin{minipage}[t]{0.45\textwidth}
    \includegraphics[width=\textwidth]{../results/ecoli_20x20_aerob_seed55_growth.pdf}
  \end{minipage}
  \begin{minipage}[t]{0.45\textwidth}
    \includegraphics[width=\textwidth]{../results/ecoli_20x20_aerob_seed55_subs.pdf}
  \end{minipage}
  \caption{Aerobic growth of ecoli core model, grid 20x20, seed=55}
\end{figure}

\subsubsection{\textit{Escherichia coli}}

\begin{figure}[h]
  \centering
  \subfigure[]{
    \begin{minipage}[t]{0.3\textwidth}
    \includegraphics[width=\textwidth]{../results/Bcoli_20x20_seed176_bac5.pdf}
  \end{minipage}
  \begin{minipage}[t]{0.3\textwidth}
    \includegraphics[width=\textwidth]{../results/Bcoli_20x20_seed176_bac35.pdf}
  \end{minipage}
  \begin{minipage}[t]{0.3\textwidth}
    \includegraphics[width=\textwidth]{../results/Bcoli_20x20_seed176_bac50.pdf}
  \end{minipage}
  }
  \subfigure[]{
  \begin{minipage}[t]{0.3\textwidth}
    \includegraphics[width=\textwidth]{../results/Bcoli_20x20_seed176_gluc5.pdf}
  \end{minipage}
  \begin{minipage}[t]{0.3\textwidth}
    \includegraphics[width=\textwidth]{../results/Bcoli_20x20_seed176_gluc35.pdf}
  \end{minipage}
  \begin{minipage}[t]{0.3\textwidth}
    \includegraphics[width=\textwidth]{../results/Bcoli_20x20_seed176_gluc50.pdf}
  \end{minipage}
  }
  \subfigure[]{
  \begin{minipage}[t]{0.3\textwidth}
    \includegraphics[width=\textwidth]{../results/Bcoli_20x20_seed176_ace5.pdf}
  \end{minipage}
  \begin{minipage}[t]{0.3\textwidth}
    \includegraphics[width=\textwidth]{../results/Bcoli_20x20_seed176_ace35.pdf}
  \end{minipage}
  \begin{minipage}[t]{0.3\textwidth}
    \includegraphics[width=\textwidth]{../results/Bcoli_20x20_seed176_ace50.pdf}
  \end{minipage}
  }
  \subfigure[]{
  \begin{minipage}[t]{0.3\textwidth}
    \includegraphics[width=\textwidth]{../results/Bcoli_20x20_seed176_co25.pdf}
  \end{minipage}
  \begin{minipage}[t]{0.3\textwidth}
    \includegraphics[width=\textwidth]{../results/Bcoli_20x20_seed176_co235.pdf}
  \end{minipage}
  \begin{minipage}[t]{0.3\textwidth}
    \includegraphics[width=\textwidth]{../results/Bcoli_20x20_seed176_co250.pdf}
  \end{minipage}
  }
  \caption{Aerobic growth of ecoli core model, grid 20x20, seed=55}
\end{figure}

\begin{figure}[h]
  \centering
  \begin{minipage}[t]{0.45\textwidth}
    \includegraphics[width=\textwidth]{../results/Bcoli_20x20_seed176_growth.pdf}
  \end{minipage}
  \begin{minipage}[t]{0.45\textwidth}
    \includegraphics[width=\textwidth]{../results/Bcoli_20x20_seed176_subs.pdf}
  \end{minipage}
  \caption{Aerobic growth of ecoli core model, grid 20x20, seed=55}
\end{figure}

\subsubsection{\textit{Methanosarcina barkeri}}

\begin{figure}[h]
  \centering
  \subfigure[]{
    \begin{minipage}[t]{0.3\textwidth}
    \includegraphics[width=\textwidth]{../results/barkeri_20x20_seed9659_bac10.pdf}
  \end{minipage}
  \begin{minipage}[t]{0.3\textwidth}
    \includegraphics[width=\textwidth]{../results/barkeri_20x20_seed9659_bac100.pdf}
  \end{minipage}
  \begin{minipage}[t]{0.3\textwidth}
    \includegraphics[width=\textwidth]{../results/barkeri_20x20_seed9659_bac130.pdf}
  \end{minipage}
  }
  \subfigure[]{
  \begin{minipage}[t]{0.3\textwidth}
    \includegraphics[width=\textwidth]{../results/barkeri_20x20_seed9659_methanol10.pdf}
  \end{minipage}
  \begin{minipage}[t]{0.3\textwidth}
    \includegraphics[width=\textwidth]{../results/barkeri_20x20_seed9659_methanol100.pdf}
  \end{minipage}
  \begin{minipage}[t]{0.3\textwidth}
    \includegraphics[width=\textwidth]{../results/barkeri_20x20_seed9659_methanol130.pdf}
  \end{minipage}
  }
  \subfigure[]{
  \begin{minipage}[t]{0.3\textwidth}
    \includegraphics[width=\textwidth]{../results/barkeri_20x20_seed9659_co210.pdf}
  \end{minipage}
  \begin{minipage}[t]{0.3\textwidth}
    \includegraphics[width=\textwidth]{../results/barkeri_20x20_seed9659_co2100.pdf}
  \end{minipage}
  \begin{minipage}[t]{0.3\textwidth}
    \includegraphics[width=\textwidth]{../results/barkeri_20x20_seed9659_co2130.pdf}
  \end{minipage}
  }
  \subfigure[]{
  \begin{minipage}[t]{0.3\textwidth}
    \includegraphics[width=\textwidth]{../results/barkeri_20x20_seed9659_meth10.pdf}
  \end{minipage}
  \begin{minipage}[t]{0.3\textwidth}
    \includegraphics[width=\textwidth]{../results/barkeri_20x20_seed9659_meth100.pdf}
  \end{minipage}
  \begin{minipage}[t]{0.3\textwidth}
    \includegraphics[width=\textwidth]{../results/barkeri_20x20_seed9659_meth130.pdf}
  \end{minipage}
  }
  \caption{Aerobic growth of ecoli core model, grid 20x20, seed=55}
\end{figure}

\begin{figure}[h]
  \centering
  \begin{minipage}[t]{0.45\textwidth}
    \includegraphics[width=\textwidth]{../results/barkeri_20x20_seed9659_growth.pdf}
  \end{minipage}
  \begin{minipage}[t]{0.45\textwidth}
    \includegraphics[width=\textwidth]{../results/barkeri_20x20_seed9659_subs.pdf}
  \end{minipage}
  \caption{Aerobic growth of ecoli core model, grid 20x20, seed=55}
\end{figure}

\subsubsection{\textit{Clostridium beijerinckii}}

\begin{figure}[h]
  \centering
  \subfigure[]{
    \begin{minipage}[t]{0.3\textwidth}
    \includegraphics[width=\textwidth]{../results/beijerinckii_20x20_seed943_bac10.pdf}
  \end{minipage}
  \begin{minipage}[t]{0.3\textwidth}
    \includegraphics[width=\textwidth]{../results/beijerinckii_20x20_seed943_bac50.pdf}
  \end{minipage}
  \begin{minipage}[t]{0.3\textwidth}
    \includegraphics[width=\textwidth]{../results/beijerinckii_20x20_seed943_bac65.pdf}
  \end{minipage}
  }
  \subfigure[]{
  \begin{minipage}[t]{0.3\textwidth}
    \includegraphics[width=\textwidth]{../results/beijerinckii_20x20_seed943_gluc10.pdf}
  \end{minipage}
  \begin{minipage}[t]{0.3\textwidth}
    \includegraphics[width=\textwidth]{../results/beijerinckii_20x20_seed943_gluc50.pdf}
  \end{minipage}
  \begin{minipage}[t]{0.3\textwidth}
    \includegraphics[width=\textwidth]{../results/beijerinckii_20x20_seed943_gluc65.pdf}
  \end{minipage}
  }
  \subfigure[]{
  \begin{minipage}[t]{0.3\textwidth}
    \includegraphics[width=\textwidth]{../results/beijerinckii_20x20_seed943_h210.pdf}
  \end{minipage}
  \begin{minipage}[t]{0.3\textwidth}
    \includegraphics[width=\textwidth]{../results/beijerinckii_20x20_seed943_h250.pdf}
  \end{minipage}
  \begin{minipage}[t]{0.3\textwidth}
    \includegraphics[width=\textwidth]{../results/beijerinckii_20x20_seed943_h265.pdf}
  \end{minipage}
  }
  \subfigure[]{
  \begin{minipage}[t]{0.3\textwidth}
    \includegraphics[width=\textwidth]{../results/beijerinckii_20x20_seed943_co210.pdf}
  \end{minipage}
  \begin{minipage}[t]{0.3\textwidth}
    \includegraphics[width=\textwidth]{../results/beijerinckii_20x20_seed943_co250.pdf}
  \end{minipage}
  \begin{minipage}[t]{0.3\textwidth}
    \includegraphics[width=\textwidth]{../results/beijerinckii_20x20_seed943_co265.pdf}
  \end{minipage}
  }
  \caption{Aerobic growth of ecoli core model, grid 20x20, seed=55}
\end{figure}

\begin{figure}[h]
  \centering
  \begin{minipage}[t]{0.45\textwidth}
    \includegraphics[width=\textwidth]{../results/beijerinckii_20x20_seed943_growth.pdf}
  \end{minipage}
  \begin{minipage}[t]{0.45\textwidth}
    \includegraphics[width=\textwidth]{../results/beijerinckii_20x20_seed943_subs.pdf}
  \end{minipage}
  \caption{Aerobic growth of ecoli core model, grid 20x20, seed=55}
\end{figure}

\subsection{Mixed communities}
\subsubsection{\textit{Escherichia coli} \& \textit{Methanosarcina barkeri}}

\begin{figure}[h]
  \centering
  \subfigure[]{
    \begin{minipage}[t]{0.3\textwidth}
    \includegraphics[width=\textwidth]{../results/barkeri_ecoli_20x20_seed4612_bac50.pdf}
  \end{minipage}
  \begin{minipage}[t]{0.3\textwidth}
    \includegraphics[width=\textwidth]{../results/barkeri_ecoli_20x20_seed4612_bac100.pdf}
  \end{minipage}
  \begin{minipage}[t]{0.3\textwidth}
    \includegraphics[width=\textwidth]{../results/barkeri_ecoli_20x20_seed4612_bac150.pdf}
  \end{minipage}
  }
  \subfigure[]{
  \begin{minipage}[t]{0.3\textwidth}
    \includegraphics[width=\textwidth]{../results/barkeri_ecoli_20x20_seed4612_gluc50.pdf}
  \end{minipage}
  \begin{minipage}[t]{0.3\textwidth}
    \includegraphics[width=\textwidth]{../results/barkeri_ecoli_20x20_seed4612_gluc100.pdf}
  \end{minipage}
  \begin{minipage}[t]{0.3\textwidth}
    \includegraphics[width=\textwidth]{../results/barkeri_ecoli_20x20_seed4612_gluc150.pdf}
  \end{minipage}
  }
  \subfigure[]{
  \begin{minipage}[t]{0.3\textwidth}
    \includegraphics[width=\textwidth]{../results/barkeri_ecoli_20x20_seed4612_ace50.pdf}
  \end{minipage}
  \begin{minipage}[t]{0.3\textwidth}
    \includegraphics[width=\textwidth]{../results/barkeri_ecoli_20x20_seed4612_ace100.pdf}
  \end{minipage}
  \begin{minipage}[t]{0.3\textwidth}
    \includegraphics[width=\textwidth]{../results/barkeri_ecoli_20x20_seed4612_ace150.pdf}
  \end{minipage}
  }
  \subfigure[]{
  \begin{minipage}[t]{0.3\textwidth}
    \includegraphics[width=\textwidth]{../results/barkeri_ecoli_20x20_seed4612_meth50.pdf}
  \end{minipage}
  \begin{minipage}[t]{0.3\textwidth}
    \includegraphics[width=\textwidth]{../results/barkeri_ecoli_20x20_seed4612_meth100.pdf}
  \end{minipage}
  \begin{minipage}[t]{0.3\textwidth}
    \includegraphics[width=\textwidth]{../results/barkeri_ecoli_20x20_seed4612_meth150.pdf}
  \end{minipage}
  }
  \caption{Aerobic growth of ecoli core model, grid 20x20, seed=55}
\end{figure}

\begin{figure}[h]
  \centering
  \begin{minipage}[t]{0.45\textwidth}
    \includegraphics[width=\textwidth]{../results/barkeri_ecoli_20x20_seed4612_growth.pdf}
  \end{minipage}
  \begin{minipage}[t]{0.45\textwidth}
    \includegraphics[width=\textwidth]{../results/barkeri_ecoli_20x20_seed4612_subs.pdf}
  \end{minipage}
  \caption{Aerobic growth of ecoli core model, grid 20x20, seed=55}
\end{figure}

\subsubsection{\textit{Escherichia coli} \& \textit{Clostridium beijerinckii}}

\begin{figure}[h]
  \centering
  \subfigure[]{
    \begin{minipage}[t]{0.3\textwidth}
    \includegraphics[width=\textwidth]{../results/ecoli_beijerinckii_20x20_seed5147_bac10.pdf}
  \end{minipage}
  \begin{minipage}[t]{0.3\textwidth}
    \includegraphics[width=\textwidth]{../results/ecoli_beijerinckii_20x20_seed5147_bac55.pdf}
  \end{minipage}
  \begin{minipage}[t]{0.3\textwidth}
    \includegraphics[width=\textwidth]{../results/ecoli_beijerinckii_20x20_seed5147_bac75.pdf}
  \end{minipage}
  }
  \subfigure[]{
  \begin{minipage}[t]{0.3\textwidth}
    \includegraphics[width=\textwidth]{../results/ecoli_beijerinckii_20x20_seed5147_gluc10.pdf}
  \end{minipage}
  \begin{minipage}[t]{0.3\textwidth}
    \includegraphics[width=\textwidth]{../results/ecoli_beijerinckii_20x20_seed5147_gluc55.pdf}
  \end{minipage}
  \begin{minipage}[t]{0.3\textwidth}
    \includegraphics[width=\textwidth]{../results/ecoli_beijerinckii_20x20_seed5147_gluc75.pdf}
  \end{minipage}
  }
  \subfigure[]{
  \begin{minipage}[t]{0.3\textwidth}
    \includegraphics[width=\textwidth]{../results/ecoli_beijerinckii_20x20_seed5147_h210.pdf}
  \end{minipage}
  \begin{minipage}[t]{0.3\textwidth}
    \includegraphics[width=\textwidth]{../results/ecoli_beijerinckii_20x20_seed5147_h255.pdf}
  \end{minipage}
  \begin{minipage}[t]{0.3\textwidth}
    \includegraphics[width=\textwidth]{../results/ecoli_beijerinckii_20x20_seed5147_h275.pdf}
  \end{minipage}
  }
  \subfigure[]{
  \begin{minipage}[t]{0.3\textwidth}
    \includegraphics[width=\textwidth]{../results/ecoli_beijerinckii_20x20_seed5147_etoh10.pdf}
  \end{minipage}
  \begin{minipage}[t]{0.3\textwidth}
    \includegraphics[width=\textwidth]{../results/ecoli_beijerinckii_20x20_seed5147_etoh55.pdf}
  \end{minipage}
  \begin{minipage}[t]{0.3\textwidth}
    \includegraphics[width=\textwidth]{../results/ecoli_beijerinckii_20x20_seed5147_etoh75.pdf}
  \end{minipage}
  }
  \caption{Aerobic growth of ecoli core model, grid 20x20, seed=55}
\end{figure}

\begin{figure}[h]

  \centering
  \begin{minipage}[t]{0.45\textwidth}
    \includegraphics[width=\textwidth]{../results/ecoli_beijerinckii_20x20_seed5147_growth.pdf}
  \end{minipage}
  \begin{minipage}[t]{0.45\textwidth}
    \includegraphics[width=\textwidth]{../results/ecoli_beijerinckii_20x20_seed5147_subs.pdf}
  \end{minipage}
  \caption{Aerobic growth of ecoli core model, grid 20x20, seed=55}
\end{figure}
\subsubsection{\textit{Clostridium beijerinckii} \& \textit{Methanosarcina barkeri}}

\begin{figure}[h]
  \centering
  \subfigure[]{
    \begin{minipage}[t]{0.3\textwidth}
    \includegraphics[width=\textwidth]{../results/barkeri_beijerinckii_20x20_seed6764_bac50.pdf}
  \end{minipage}
  \begin{minipage}[t]{0.3\textwidth}
    \includegraphics[width=\textwidth]{../results/barkeri_beijerinckii_20x20_seed6764_bac100.pdf}
  \end{minipage}
  \begin{minipage}[t]{0.3\textwidth}
    \includegraphics[width=\textwidth]{../results/barkeri_beijerinckii_20x20_seed6764_bac150.pdf}
  \end{minipage}
  }
  \subfigure[]{
  \begin{minipage}[t]{0.3\textwidth}
    \includegraphics[width=\textwidth]{../results/barkeri_beijerinckii_20x20_seed6764_gluc50.pdf}
  \end{minipage}
  \begin{minipage}[t]{0.3\textwidth}
    \includegraphics[width=\textwidth]{../results/barkeri_beijerinckii_20x20_seed6764_gluc100.pdf}
  \end{minipage}
  \begin{minipage}[t]{0.3\textwidth}
    \includegraphics[width=\textwidth]{../results/barkeri_beijerinckii_20x20_seed6764_gluc150.pdf}
  \end{minipage}
  }
  \subfigure[]{
  \begin{minipage}[t]{0.3\textwidth}
    \includegraphics[width=\textwidth]{../results/barkeri_beijerinckii_20x20_seed6764_h250.pdf}
  \end{minipage}
  \begin{minipage}[t]{0.3\textwidth}
    \includegraphics[width=\textwidth]{../results/barkeri_beijerinckii_20x20_seed6764_h2100.pdf}
  \end{minipage}
  \begin{minipage}[t]{0.3\textwidth}
    \includegraphics[width=\textwidth]{../results/barkeri_beijerinckii_20x20_seed6764_h2150.pdf}
  \end{minipage}
  }
  \subfigure[]{
  \begin{minipage}[t]{0.3\textwidth}
    \includegraphics[width=\textwidth]{../results/barkeri_beijerinckii_20x20_seed6764_meth50.pdf}
  \end{minipage}
  \begin{minipage}[t]{0.3\textwidth}
    \includegraphics[width=\textwidth]{../results/barkeri_beijerinckii_20x20_seed6764_meth100.pdf}
  \end{minipage}
  \begin{minipage}[t]{0.3\textwidth}
    \includegraphics[width=\textwidth]{../results/barkeri_beijerinckii_20x20_seed6764_meth150.pdf}
  \end{minipage}
  }
  \caption{Aerobic growth of ecoli core model, grid 20x20, seed=55}
\end{figure}

\begin{figure}[h]
  \centering
  \begin{minipage}[t]{0.45\textwidth}
    \includegraphics[width=\textwidth]{../results/barkeri_beijerinckii_20x20_seed6764_growth.pdf}
  \end{minipage}
  \begin{minipage}[t]{0.45\textwidth}
    \includegraphics[width=\textwidth]{../results/barkeri_beijerinckii_20x20_seed6764_subs.pdf}
  \end{minipage}
  \caption{Aerobic growth of ecoli core model, grid 20x20, seed=55}
\end{figure}

%\subsection{Spatial and time-wise heterogeneity}
%\subsubsection{Substrate gradients}
%\subsubsection{Delayed bacterial input}


\section{Discussion}
\subsection{Movement and diffusion}
In agent based modeling there are two different rules for updating:
A Synchronous mode updates all cells simultanously, i.e. local changes are stored in a temporary copy and only after computation of all cells this changes will be efficacious.
Contrary to this, in a asynchronous mode changes will be made immediately (\cite{Matthies2002} p. 92).\\
Implementing a simple, naive diffusion model, which interchanges states between neighboured cells, asynchronous updating with randomly choosen cells is preferred because i) synchronouse updating violates conservation laws and ii) nonrandomly asynchronousy is causing a biased diffusion direction \cite{Bandman1999}.

Further work could be done by implementing more sofisticated diffusion models like block-rotation \cite{Bandman1999} or the discrete diffusion model by Grajdeanu \cite{Grajdeanu2007}.
\subsection{Population models of single organisms}
\subsection{Synthrophy in mixed communities}
\subsection{From heterogeneity to biofilms}
\subsection{Conclusions \& outlook}



\printbibliography 

\end{document}


\begin{items}
  \item
\end{items}
\begin{enumerate}
  \item
\end{enumerate}
